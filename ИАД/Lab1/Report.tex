\documentclass[a4paper,14pt]{extarticle}
\usepackage{../../tex-shared/preamble}

\renewcommand{\mylabnumber}{1}
\renewcommand{\mylabtitle}{Исследование возможностей языка
R для статического анализа данных}
\renewcommand{\mysubject}{Интеллектуальный анализ данных}
\renewcommand{\mylecturer}{Сырых О.А.}

\begin{document}
\begin{titlepage}
    
    \thispagestyle{empty}
    
    \begin{center}
        
        Министерство науки и Высшего образования Российской Федерации \\
        Севастопольский государственный университет \\
        Кафедра ИС
        
        \vfill

        Отчет \\
        по лабораторной работе №\mylabnumber \\
        \enquote{\mylabtitle} \\
        по дисциплине \\
        \enquote{\MakeTextUppercase{\mysubject}}

    \end{center}

    \vspace{1cm}

    \noindent\hspace{7.5cm} Выполнил студент группы ИС/б-17-2-о \\
    \null\hspace{7.5cm} Горбенко К. Н. \\
    \null\hspace{7.5cm} Проверил \\
    \null\hspace{7.5cm} \mylecturer

    \vfill

    \begin{center}
        Севастополь \\
        2020
    \end{center}

\end{titlepage}

\section{Цель работы}
\begin{itemize}
    \item изучить основные особенности языка R;
    \item исследовать возможности языка R для работы с графикой.
\end{itemize}

\section{Задание на работу}
\begin{enumerate}
    \item Установить R на ПК.
    \item Установить RStudio - инсталлятор скачать с официального сайта проекта.
    \item Ознакомиться с кратким руководством пользователя RStudio.
    \item Исследовать команду \code{\enquote{demo()}}, полученные результаты вставить в отчет.
    \item Исследовать основные функции и команды языка R, представленные в данной
          лабораторной работе, полученные результаты вставить в отчет.
    \item Ответить на контрольные вопросы.
\end{enumerate}

\section{Ход работы}
Используем консоль \code{R.exe}. При входе получаем следующий текст:

\begin{lstlisting}
R version 3.6.2 (2019-12-12) -- "Dark and Stormy 
Copyright (C) 2019 The R Foundation for Statistic
Platform: x86_64-w64-mingw32/x64 (64-bit)

R is free software and comes with ABSOLUTELY NO W
You are welcome to redistribute it under certain 
Type 'license()' or 'licence()' for distribution 

    Natural language support but running in an Engl
'citation()' on how to cite R or R packages in publications.

Type 'demo()' for some demos, 'help()' for on-line help, or
'help.start()' for an HTML browser interface to help.
Type 'q()' to quit R.
\end{lstlisting}

Выполним команду \code{demo()}. Результат выполнения команды представлен
на рисунке \ref{fig:demo}.
\begin{figure}[H]
    \centering
    \includegraphics[width=.8\linewidth]{demo}
    \caption{Результат выполнения команды \code{demo()}}
    \label{fig:demo}
\end{figure}

Выполним команду \code{help(demo)}. Результат ее выполнения изображен на рисунке \ref{fig:help-demo}
\begin{figure}[H]
    \centering
    \includegraphics[width=.8\linewidth]{help-demo}
    \caption{Результат выполнения команды \code{help(demo)}}
    \label{fig:help-demo}
\end{figure}

Выполним операции над векторами:
\begin{lstlisting}
> v1 <- c(1, 2, 3, 4, 5) * 3
> v1
[1]  3  6  9 12 15
> v1[1:3]
[1] 3 6 9
> v2 = rep(15, 4)
> v2
[1] 15 15 15 15
> length(v2)
[1] 4
\end{lstlisting}

Выполним операции над матрицами:
\begin{lstlisting}
> mat1 <- matrix (data=1, nrow=3, ncol=3)
> mat1
    [,1] [,2] [,3]
[1,]    1    1    1
[2,]    1    1    1
[3,]    1    1    1
> is.matrix(mat1)
[1] TRUE
> dim(mat1)
[1] 3 3
> dim(mat1)
[1] 3 3
\end{lstlisting}

Работа с графикой:
\begin{lstlisting}
> x <- c(1, 2, 3, 4, 5)
> y <- c(1, 2, 3, 4, 5)
> plot(x, y, type="l")
> plot(atan, -2 * pi, 2* pi)
\end{lstlisting}

Результат выполнения программы:
\begin{figure}[H]
    \centering
    \subfloat[y = x]{\includegraphics[width=.4\linewidth]{y=x}}
    \hspace{.15\linewidth}
    \subfloat[y = arctan(x)]{\includegraphics[width=.4\linewidth]{atan}}
    \caption{Демонстация работы с графикой}
\end{figure}

\section{Контрольные вопросы}
\begin{enumerate}
    \item Особенности языка R.
    
    R - язык программирования, ориентированный на статическую обработку данных.
    Он является интерпретируемым языком, основным инструментом для работы с
    которым является командный интерпретатор \code{R.exe}.

    Основной программной единицей в языке являются объекты (скалярные базовые
    значения отсутствуют).

    Для присваивания используется следующий оператор: \code{<-}. Язык регистрозависим.

    \item Для получения подробной информации о функции используется команда \code{help}.
    \item Команда \code{Round}
\end{enumerate}
\end{document}