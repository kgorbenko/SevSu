\documentclass[a4paper,14pt]{extarticle}
\usepackage{../../tex-shared/report-layout}

\renewcommand{\mylabnumber}{4}
\renewcommand{\mylabtitle}{Кластерный анализ. Основные этапы и задачи кластерного анализа данных}
\renewcommand{\mysubject}{Интеллектуальный анализ данных}
\renewcommand{\mylecturer}{Сырых О.А.}

\begin{document}
\begin{titlepage}
    
    \thispagestyle{empty}
    
    \begin{center}
        
        Министерство науки и Высшего образования Российской Федерации \\
        Севастопольский государственный университет \\
        Кафедра ИС
        
        \vfill

        Отчет \\
        по лабораторной работе №\mylabnumber \\
        \enquote{\mylabtitle} \\
        по дисциплине \\
        \enquote{\MakeTextUppercase{\mysubject}}

    \end{center}

    \vspace{1cm}

    \noindent\hspace{7.5cm} Выполнил студент группы ИС/б-17-2-о \\
    \null\hspace{7.5cm} Горбенко К. Н. \\
    \null\hspace{7.5cm} Проверил \\
    \null\hspace{7.5cm} \mylecturer

    \vfill

    \begin{center}
        Севастополь \\
        2020
    \end{center}

\end{titlepage}

\section{Цель работы}
\begin{itemize}
    \item закрепить теоретические знания и приобрести практические навыки в
          проведении кластерного анализа по экспериментальным данным;
    \item исследовать возможности языка R для проведения кластерного анализа.
\end{itemize}

\section{Задание на работу}
\begin{enumerate}
    \item Создать файл с исходными данными.
    \item Провести кластерный анализ экспериментальных данных.
    \item Проведя процедуру кластеризации несколько раз при различных значениях
          числа кластеров (от 2-х до 10 кластеров), необходимо выбрать лучшую
          группировку в смысле критерия минимума отношений средних внутри
          кластерных и меж кластерных расстояний. Полученные результаты оформите
          в виде таблицы. Изобразить графически значения данного показателя
          качества классификации. Для этого построить диаграмму, на которой по
          оси Х – количество кластеров, по оси Y – значения показателя J.
\end{enumerate}

\section{Ход работы}
\subsection{Кластерный анализ}
Выполним иерархический кластерный анализ. Загрузим данные о количестве заражений
коронавирусом \ref{fig:data-set}.

\begin{figure}[H]
    \centering
    \includegraphics[width=.9\linewidth]{data-set}
    \caption{Загруженные данные}
    \label{fig:data-set}
\end{figure}

\begin{itemize}
    \item total\_cases\_by\_1m - количество заражений на 1 миллион населения;
    \item total\_death\_by\_1m - количество смертей на 1 миллион населения;
    \item tests\_by\_1m - количество тестов на 1 миллион населения;
    \item health\_index - индекс развитости системы здравоохранения;
    \item stringency\_index - индекс реакции властей на пандемию.
\end{itemize}

Функция кластерного анализа в R:

\code{kmeans(x, centers, iter.шах=10, nstart=1,\\algorithm=c("Hartigan-Wong", "Lloyd", "Forgy", "MacQueen"))}

Выполним разбиение на два кластера по переменным \enquote{Индекс развитости системы здравоохранения} и
\enquote{Индекс реакции властей на пандемию}. Результат разбиения изображен на рисунке
\ref{fig:2-clusters-graph}.

\begin{figure}[H]
    \centering
    \includegraphics[width=.65\linewidth]{2-clusters-graph}
    \caption{Разбиение данных на 2 кластера}
    \label{fig:2-clusters-graph}
\end{figure}
\pagebreak

Результаты разбиения представлены на рисунке \ref{fig:2-clusters-R}.

\begin{figure}[H]
    \centering
    \includegraphics[width=.8\linewidth]{2-clusters-R}
    \caption{Результаты разбиения}
    \label{fig:2-clusters-R}
\end{figure}

\begin{itemize}
    \item первый кластер содержит 32 элемента, второй – 4;
    \item сумма квадратов расстояний внутри кластера: 1 – 2487.778, 2 – 1022.395;
    \item общая сумма квадратов расстояний внутри кластеров: 3510.174;
    \item сумма квадратов расстояний между кластерами – 5394.984.
\end{itemize}

Для выбора лучшей группировки в смысле критерия минимума отношений средних
внутри кластерных и меж кластерных расстояний было проведено деление на 2 – 10
кластеров и заполнена таблица в MS Excel.

\begin{figure}[H]
    \centering
    \includegraphics[width=.6\linewidth]{excel-table}
    \caption{Расчет численного показателя меры качества классификации}
    \label{fig:excel-table}
\end{figure}

Значения данного показателя качества классификации представлено графически на
рис 4.  Для этого построена диаграмма, на которой по оси Х – количество
кластеров, по оси Y – значения показателя J.

\begin{figure}[H]
    \centering
    \includegraphics[width=.5\linewidth]{excel-graph}
    \caption{Диаграмма численной меры качества классификации}
    \label{fig:excel-graph}
\end{figure}

В соответствии с этим критерием оптимальным разбиением экспериментальных данных
является разбиение на 3 кластера.

\subsection{Иерархический анализ}
Был проведен иерархический анализ методом Уорда:

\begin{figure}[H]
    \centering
    \includegraphics[width=\linewidth]{hierarchial-ward}
    \caption{Иерархический анализ методом Уорда}
    \label{fig:hierarchial-ward}
\end{figure}

Результатом анализа является количество элементов в каждом из кластеров:

\begin{lstlisting}
Rcmdr>  summary(as.factor(cutree(HClust.3, k = 3))) # Cluster Sizes
1  2  3
20 12  4
\end{lstlisting}

Был проведен иерархический анализ методом простой связи:

\begin{figure}[H]
    \centering
    \includegraphics[width=\linewidth]{hierarchial-single-link}
    \caption{Иерархический анализ методом простой связи}
    \label{fig:hierarchial-single-link}
\end{figure}

Результатом анализа является количество элементов в каждом из кластеров:

\begin{lstlisting}
Rcmdr>  summary(as.factor(cutree(sl, k = 3))) # Cluster Sizes
1  2  3
32  3  1
\end{lstlisting}

\section*{Выводы}
В ходе лабораторной работы провели многомерный анализ данных. Для этого
использовали кластеризацию, которая предназначена для разбиения совокупности
объектов на однородные группы (кластеры или классы). Если данные выборки
представить как точки в признаковом пространстве, то задача кластеризации
сводится к определению "сгущений точек". Целью кластеризации является поиск
существующих структур. Кластерный анализ позволяет сокращать размерность данных,
делать ее наглядной. Провели иерархический анализ с использованием k-средних.
Данный алгоритм является наиболее распространённым.

\end{document}