\documentclass[a4paper,14pt]{extarticle}

\usepackage{ucs}                                                                                                                   
\usepackage[utf8x]{inputenc}                                                                                                       
\usepackage[english,russian]{babel}
\usepackage[T2A]{fontenc}
\usepackage{extsizes}
\usepackage{tempora}
\usepackage[left=30mm, top=20mm, right=15mm, bottom=20mm, headheight=5pt]{geometry}
\usepackage{fancyhdr}
\usepackage{titling}
\usepackage{titlesec}
\usepackage{textcase}
\usepackage{indentfirst}
\usepackage{graphicx}
\usepackage{float}
\usepackage{caption}
\usepackage{listings}
\usepackage{color}
\usepackage{enumitem}

\definecolor{cyan}{rgb}{0.0,0.6,0.6} % colors defined for listings with "listings" package
\definecolor{red}{rgb}{0.6,0,0}
\definecolor{green}{rgb}{0,0.8,0}

\graphicspath{ {./images/} }

\newcommand{\mylabnumber}{1}
\newcommand{\mylabtitle}{Встроенные типы данных в C\#. Массивы. Строки. Регулярные выражения}
\newcommand{\mysubject}{Платформа .NET}
\newcommand{\mylecturer}{ст. преп. Забаштанский А.К.}

\renewcommand{\baselinestretch}{1.25} % Sets basic line stretch
\renewcommand{\headrulewidth}{0pt} % Remove horizontal line below header in fancyhdr

\addto\captionsrussian{
    \renewcommand{\figurename}{Рисунок} % Set a default picture caption
    \renewcommand{\tablename}{Таблица} % Set a default table caption
}

\captionsetup[table]{singlelinecheck=false} % To make a table caption appear left-aligned

\pagestyle{fancy}
\lhead{} \rhead{} \cfoot{} % Setting empty headers
\chead{\thepage} % Sets central header page numbering

\setlength{\parindent}{1.25cm}
\setlength{\parskip}{8pt}

\titleformat{\section}[hang]{\large \centering \bfseries}{\thesection}{0.5em}{\MakeTextUppercase} % Format section style
\titlespacing{\section}{\parindent}{1em}{0em}

\titleformat{\subsection}[hang]{\bfseries}{\thesubsection}{0.5em}{} % Format subsection style
\titlespacing{\subsection}{\parindent}{1em}{0em} % Format subsection indentations

\titleformat{\subsubsection}[hang]{\normalfont}{\thesubsubsection}{0.5em}{} % Format subsubsection style
\titlespacing{\subsubsection}{\parindent}{1em}{0em} % Format subsubsection indentations

\setlist[enumerate]{wide=\parindent, leftmargin=0pt, topsep=0pt, itemsep=0pt, partopsep=0pt, parsep=0pt} % Format enumerate style
\setlist[itemize]{wide=\parindent, leftmargin=0pt, topsep=0pt, itemsep=0pt, partopsep=0pt, parsep=0pt} % Format itemize style

\begin{document}

    \lstset{ % "listings package configuration"
        basicstyle=\footnotesize\ttfamily,
        numbersep=5pt,
        tabsize=4,
        gobble=8,
        extendedchars=\true,
        keepspaces=\true,
        numbers=left,
        keywordstyle=\color{cyan},
        stringstyle=\color{red}\ttfamily,
        commentstyle=\color{green},
        showstringspaces=\false
    }

    \begin{titlepage}
        
        \thispagestyle{empty}
        
        \begin{center}
            
            Министерство науки и высшего образования Российской Федерации \\
            Севастопольский государственный университет \\
            Кафедра ИС
            
            \vfill
            \large{
                Отчет \\
                по лабораторной работе №\mylabnumber \\
                "\mylabtitle" \\
                по дисциплине \\
                \MakeTextUppercase{\mysubject}
            }

        \end{center}

        \vspace{1cm}

        \noindent\hspace{7.5cm} Выполнил студент группы ИС/б-22о \\
        \null\hspace{7.5cm} Горбенко К. Н. \\
        \null\hspace{7.5cm} Проверил \\
        \null\hspace{7.5cm} \mylecturer

        \vfill

        \begin{center}
            Севастополь \\
            2019
        \end{center}

    \end{titlepage}

    % ############################################################################
    % ------------------------------ Document start ------------------------------
    % ############################################################################

    \section{Цель работы}

    \begin{itemize}
        
        \item изучить классификацию типов данных и отличительные 
              особенности синтаксических конструкций языка C\# от C++;

        \item изучить базовые типы: Array, String, StringBuilder, а также
              средства стандартного ввода/вывода и возможности форматирования
              вывода;

        \item получить понятие о регулярных выражениях и их применении
              для поиска, замены и разбиения текста на синтаксические лексемы.

    \end{itemize}

    \section{Задание к лабораторной работе}

    Для \textbf{варианта № 3} заданы следующие задания:
    
    \begin{enumerate}
        
        \item Проработать примеры программ 1-8, данные в теоретических сведениях.
              Создать на их основе программы. Получить результаты работы программ
              и уметь их объяснить. Внести их в отчет по работе с комментариями.

        \item Найти номер столбца двухмерного массива целых чисел, для которого
              среднеарифметическое его элементов максимально.
              
        \item Создать программу, которая будет вводить строку в переменную String.
              Найти в ней те слова, которые начинаются и заканчиваются одной и 
              той же буквой.

        \item Задан текст. Определить, является ли он текстом на английском языке.

    \end{enumerate}
    
    \section{Ход работы}

    \subsection{Примеры программ на C\#}

    \subsubsection{Hello, World!}

    \begin{lstlisting}[language={[Sharp]C}]
        namespace ConsoleHello
        { 
            class Program
            {	
                static void Main(string[] args)
                { 
                    Console.WriteLine("Enter your name"); 
                    string name;
                    name = Console.ReadLine();
                    if (name == "") Console.WriteLine("Hello, world!"); 
                    else Console.WriteLine("Hello, " + name + "!");
                }
            }
        }
    \end{lstlisting}

\end{document}