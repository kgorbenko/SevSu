\documentclass[a4paper,14pt]{extarticle}

\usepackage{ucs}                                                                                                                   
\usepackage[utf8x]{inputenc}                                                                                                       
\usepackage[english,russian]{babel}
\usepackage[T2A]{fontenc}
\usepackage{extsizes}
\usepackage{tempora}
\usepackage[left=25mm, top=20mm, right=10mm, bottom=20mm, headheight=5pt]{geometry}
\usepackage{fancyhdr}
\usepackage{titling}
\usepackage{titlesec}
\usepackage{textcase}
\usepackage{indentfirst}
\usepackage{graphicx}
\usepackage{float}
\usepackage[labelsep=endash]{caption}
\usepackage{listings}
\usepackage{color}
\usepackage{enumitem}
\usepackage[font=normalsize]{subfig}
\usepackage{csquotes}
\usepackage{amsmath}

\graphicspath{ {./images/} }

\newcommand{\mylabnumber}{2}
\newcommand{\mylabtitle}{Исследование типов данных, определяемые пользователем. Наследование. Обработка исключений в C\#}
\newcommand{\mysubject}{Платформа .NET}
\newcommand{\mylecturer}{Забаштанский А.К.}

\renewcommand{\baselinestretch}{1.25} % Sets basic line stretch
\renewcommand{\headrulewidth}{0pt} % Remove horizontal line below header in fancyhdr

\addto\captionsrussian{
    \renewcommand{\figurename}{Рисунок} % Set a default picture caption
    \renewcommand{\tablename}{Таблица} % Set a default table caption
}

\captionsetup[table]{singlelinecheck=false} % To make a table caption appear left-aligned

\pagestyle{fancy}
\lhead{} \rhead{} \cfoot{} % Setting empty headers
\chead{\thepage} % Sets central header page numbering

\setlength{\parindent}{1.25cm}
\setlength{\parskip}{8pt}

 % Format section style and indentations
\titleformat{\section}[hang]{\large \centering \bfseries}{\thesection}{0.5em}{\MakeTextUppercase}
\titlespacing{\section}{\parindent}{1em}{0em}

% Format subsection style and indentations
\titleformat{\subsection}[hang]{\bfseries}{\thesubsection}{0.5em}{}
\titlespacing{\subsection}{\parindent}{1em}{0em}

% Format subsubsection style and indentations
\titleformat{\subsubsection}[hang]{\normalfont}{\thesubsubsection}{0.5em}{}
\titlespacing{\subsubsection}{\parindent}{1em}{0em}

% Format enumerate style with "enumitem" package
\setlist[enumerate, 1]{wide=\parindent,leftmargin=0pt,topsep=0pt,
    itemsep=0pt,partopsep=0pt,parsep=0pt}
\setlist[enumerate, 2]{wide=2\parindent,label=\alph*.,leftmargin=1.25cm,topsep=0pt,
    itemsep=0pt,partopsep=0pt,parsep=0pt}
\setlist[enumerate, 3]{wide=3\parindent,label=\alph*.,leftmargin=2.5cm,topsep=0pt,
    itemsep=0pt,partopsep=0pt,parsep=0pt}

% Format itemize style with "enumitem" package
\setlist[itemize, 1]{wide=\parindent,leftmargin=0pt,topsep=0pt,itemsep=0pt,
    partopsep=0pt,parsep=0pt}
\setlist[itemize, 2]{wide=2\parindent,leftmargin=1.25cm,topsep=0pt,itemsep=0pt,
    partopsep=0pt,parsep=0pt}
\setlist[itemize, 3]{wide=3\parindent,leftmargin=2.5cm,topsep=0pt,itemsep=0pt,
    partopsep=0pt,parsep=0pt}

\captionsetup[figure]{justification=centering}

\begin{document}

    \lstset{ % "listings package configuration"
        basicstyle=\footnotesize\ttfamily,
        breaklines=true,
        numbersep=5pt,
        tabsize=4,
        gobble=8,
        extendedchars=\true,
        keepspaces=\true,
        numbers=left,
        stringstyle=\ttfamily,
        showstringspaces=\false
    }

    % ############################################################################
    % -------------------------------- Title page --------------------------------
    % ############################################################################

    \begin{titlepage}
        
        \thispagestyle{empty}
        
        \begin{center}
            
            Министерство науки и высшего образования Российской Федерации \\
            Севастопольский государственный университет \\
            Кафедра ИС
            
            \vfill

            Отчет \\
            по лабораторной работе №\mylabnumber \\
            \enquote{\mylabtitle} \\
            по дисциплине \\
            \enquote{\MakeTextUppercase{\mysubject}}

        \end{center}

        \vspace{1cm}

        \noindent\hspace{7.5cm} Выполнил студент группы ИС/б-17-2-о \\
        \null\hspace{7.5cm} Горбенко К. Н. \\
        \null\hspace{7.5cm} Проверил \\
        \null\hspace{7.5cm} \mylecturer

        \vfill

        \begin{center}
            Севастополь \\
            2019
        \end{center}

    \end{titlepage}

    % ############################################################################
    % ------------------------------ Document start ------------------------------
    % ############################################################################

    \section{Цель работы}
    \begin{enumerate}
        \item Познакомиться с пользовательскими типами данных в языке С\#:
              структура и перечисление.
        \item Ознакомиться со структурой класса, его созданием и использованием,
              описанием членов класса: полей, свойств, инициализации объектов
              класса с помощью конструкторов.
        \item Изучить механизм создания иерархий классов в С\# и применение
              интерфейсов при наследовании.
        \item Изучить механизм генерации и обработки исключений.
    \end{enumerate}

    \section{Задание на лабораторную работу}
    \begin{enumerate}
        \item Проработать примеры программ 1-6, данные в теоретических сведениях.
              Создать на их основе программы. Получить результаты работы программ
              и уметь их объяснить. Внести в отчет с комментариями.
        \item Для заданной структуры данных разработать абстрактный класс и класс-наследник.
              В классе реализовать несколько конструкторов. Создать методы, работающие с
              полями класса. Часть из них должны быть виртуальными. Добавить методы-свойства
              и индексаторы.
        \item Разработать интерфейсные классы, добавляющие некоторые методы в класс-потомок.
              Изучить причины возникновения коллизий имен при наследовании и способы их устранения.
        \item Разработать классы исключительных ситуаций и применить их для обработки возникающих исключений.
        \item Написать демонстрационную программу.
    \end{enumerate}

    Описание данных пользовательских типов:

    \textbf{ПЕЧАТНОЕ ИЗДАНИЕ}: название, ФИО автора, стоимость, оглавление.
    
    \section{Ход работы}
    Рассмотрим примеры программ 1-5.
    \subsection{Программы}
    \begin{enumerate}
        \item Программа демонстрирует использование перечислений со значениями по умолчанию
              и установленными значениями. Кроме того, программа демонстрирует использование
              структур. При этом, не показано свойство структур как значимых классов, что, по
              моему мнению, является более важным свойством, чем наличие конструкторов, о чем
              говорят методические указания.
        \item Программа показывает возможности использования свойств как способа доступа к полям
              класса. При этом упрощается регулирование доступа и уменьшается количество 
              необходимого для этого программного кода.
        \item Программа демонстрирует использование индексаторов для созданных классов. При этом
              возможно добавление дополнительной логики к уже существующим классам с индексаторами.
        \item Программа демонстрирует возможности объектного программирования в языке C\#. Метод
              базового класса переопределяется методом класса-наследника. Кроме того, класс-наследник
              добавляет свои собственные методы к уже унаследованным.
        \item Программа показывает способы устранения коллизий имен при наследовании и реализации
              интерфейсов. Для устранения коллизий необходимо явно указывать интерфейс, для которого
              определяется метод.
    \end{enumerate}
    
    \subsection{Реализация программ в соответствии с вариантом}
    Напишем класс, соответствующий печатному изданию:
    \begin{lstlisting}[language={[Sharp]C}]
        public abstract class PrintedEdition
        {
            public string Name   { get; set; }
            public string Author { get; set; }
            public double Price  { get; set; }
            public IDictionary<string, string> Contents { get; set; }
            public PrintedEdition() { }
            public PrintedEdition(string name)
            {
                Name = name ?? throw new ArgumentNullException("name was null.");
            } 
            public PrintedEdition(string name, string author, double price, IDictionary<string, string> contents)
            {
                Name = name ?? throw new ArgumentNullException(nameof(name));
                Author = author ?? throw new ArgumentNullException(nameof(author));
                Contents = contents ?? throw new ArgumentNullException(nameof(contents));
                Price = price;
            }
            public string this[string chapter]
            {
                get 
                {
                    if (chapter == null)
                    {
                        throw new ArgumentNullException(nameof(chapter));
                    }
                    return Contents.ContainsKey(chapter) ? Contents[chapter] : throw new ChapterNotFoundExeption($"The key \"{chapter}\" was not found");
                }
                set
                {
                    if (value != null)
                    {
                        Contents[chapter] = value;
                    }
                }
            }

            public abstract string Print();
        }       
    \end{lstlisting}
    Абстрактный класс содержит свойства, конструкторы, индексатор и абстрактный метод.
    Кроме того, его индексатор выбрасывает объявленное исключение, представленное ниже:

    \begin{lstlisting}[language={[Sharp]C}]
        public class ChapterNotFoundExeption : KeyNotFoundException
        {
            public ChapterNotFoundExeption() { }
            public ChapterNotFoundExeption(string name) : base(name) { }
            public ChapterNotFoundExeption(string name, Exception inner) { }
        }
    \end{lstlisting}

    Напишем теперь его наследника:

    \begin{lstlisting}[language={[Sharp]C}]
        public class Book : PrintedEdition, IPublishable, INewPubishable
        {
            public string TitleImage { get; set; }
            public Book() { }
            public Book(string name) : base(name) { } 

            public Book(string name, string author, double price, IDictionary<string, string> contents, string titleImage) : base(name, author, price, contents) 
            {
                TitleImage = titleImage ?? throw new ArgumentNullException(nameof(titleImage));
            }
            public override string Print()
            {
                var builder = new StringBuilder();
                foreach (var chapter in Contents)
                    builder.Append($"Chapter {chapter.Key}\n\n{chapter.Value}\n\n");
                return builder.ToString();
            }

            public Book RePublish()
            {
                return new Book()
                {
                    Name = this.Name + "new edition",
                    Author = this.Author,
                    Price = this.Price,
                    Contents = this.Contents,
                    TitleImage = this.TitleImage
                };
            }

            Book IPublishable.RePublish()
            {
                return new Book() {
                    Name = "IPublishable"
                };
            }

            Book INewPubishable.RePublish()
            {
                return new Book() {
                    Name = "INewPublishable"
                };
            }
        }
    \end{lstlisting}
    Данный класс является наследником класса PrintedEdition и реализует интерфейсы
    IPublishable и INewPubishable, приведенные далее. Его конструкторы вызывают
    конструкторы базового класса, а метод Print определяет данный метод базового
    класса.

    Кроме того, метод RePublish реализуется для разных интерфейсов по-разному.

    Интерфейсы IPublishable и INewPubishable:

    \begin{lstlisting}[language={[Sharp]C}]
        public interface IPublishable
        {
            Book RePublish();
        }

        public interface INewPubishable
        {
            Book RePublish();
        }
    \end{lstlisting}

    Демонстрационная программа:

    \begin{lstlisting}[language={[Sharp]C}]
        public class Program
        {
            public static void Main(string[] args)
            {
                var contents = new Dictionary<string, string>()
                {
                    { "First", "Some content here" } 
                };
                var book = new Book("name", "me", 500, contents, "image.jpg");

                try 
                {
                    Console.WriteLine(book.RePublish().Name);
                    Console.WriteLine((book as IPublishable).RePublish().Name);
                    Console.WriteLine((book as INewPubishable).RePublish().Name);
                    var chapter = book["not existing chapter"];
                }
                catch (ChapterNotFoundExeption e) 
                {
                    Console.WriteLine(e.Message);
                }
            }
        }
    \end{lstlisting}

    \section{Выводы}

    В ходе лабораторной работы были рассмотрены механизмы ООП в языке C\#.
    Поблемы, возникающие наследовании и реализации интерфейсов - коллизии имен, 
    решаются явным указанием интерфейса. Собственные исключения описываются
    как классы, унаследованные от любого другого класса исключения.

    Классы от структур отличаются тем, что классы передаются по значению,
    а структуры по ссылке.

\end{document}