\documentclass[a4paper,14pt]{extarticle}
\usepackage{../../tex-shared/preamble}

\renewcommand{\mylabnumber}{2}
\renewcommand{\mylabtitle}{Исследование объектной
модели документа (DOM) и системы событий JavaScript}
\renewcommand{\mysubject}{Веб-технологии}
\renewcommand{\mylecturer}{Дрозин А.Ю.}

\begin{document}
\begin{titlepage}
    
    \thispagestyle{empty}
    
    \begin{center}
        
        Министерство науки и Высшего образования Российской Федерации \\
        Севастопольский государственный университет \\
        Кафедра ИС
        
        \vfill

        Отчет \\
        по лабораторной работе №\mylabnumber \\
        \enquote{\mylabtitle} \\
        по дисциплине \\
        \enquote{\MakeTextUppercase{\mysubject}}

    \end{center}

    \vspace{1cm}

    \noindent\hspace{7.5cm} Выполнил студент группы ИС/б-17-2-о \\
    \null\hspace{7.5cm} Горбенко К. Н. \\
    \null\hspace{7.5cm} Проверил \\
    \null\hspace{7.5cm} \mylecturer

    \vfill

    \begin{center}
        Севастополь \\
        2020
    \end{center}

\end{titlepage}

\section{Цель работы}
Изучить динамическую объектную модель документа, предоставляемую
стандартом DOM и систему событий языка JavaScript, возможность
хранения данных на стороне клиента. Приобрести практические навыки
работы с событиями JavaScript, деревом документа, Local Storage и
Cookies.

\section{Задание на работу}
\begin{enumerate}
    \item Реализовать отображение в области меню сайта текущих даты и
          времени (обновление времени 1 раз в секунду). 
    \item На странице «Контакт» добавить поле «Дата рождения», для
          которого реализовать всплывающий снизу элемент «календарь».
    \item Реализовать динамическую проверку корректности заполнения
          пользователем формы на странице «Контакт» таким образом, чтобы при
          потере фокуса заполняемого поля осуществлялась проверка корректности
          его заполнения. В случае если поле заполнено корректно, оно должно быть
          подсвечено зеленым цветом, иначе оно должно быть подсвечено красным,
          а после данного поля должна появиться надпись, поясняющая характер
          ошибки. После исправления пользователем ошибки, надпись должна исчезнуть. Если все поля формы заполнены корректно должна стать активной кнопка «Отправить».
    \item Реализовать открытие в динамически формируемом новом окне
          (блоке DIV) соответствующих больших фото при щелчке мыши
          по маленьким фото на странице «Фотоальбом».
    \item Добавить страницу «История просмотра». На данной странице
          реализовать отображение двух таблиц: \enquote{История текущего сеанса},
          \enquote{История за все время}.
\end{enumerate}

\section{Ход работы}

\end{document}