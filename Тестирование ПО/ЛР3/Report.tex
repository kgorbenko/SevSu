\documentclass[a4paper,14pt]{extarticle}

\usepackage{ucs}                                                                                                                   
\usepackage[utf8x]{inputenc}                                                                                                       
\usepackage[english,russian]{babel}
\usepackage[T2A]{fontenc}
\usepackage{extsizes}
\usepackage{tempora}
\usepackage[left=25mm, top=20mm, right=10mm, bottom=20mm, headheight=5pt]{geometry}
\usepackage{fancyhdr}
\usepackage{titling}
\usepackage{titlesec}
\usepackage{textcase}
\usepackage{indentfirst}
\usepackage{graphicx}
\usepackage{float}
\usepackage[labelsep=endash]{caption}
\usepackage{listings}
\usepackage{color}
\usepackage{enumitem}
\usepackage[font=normalsize]{subfig}
\usepackage{csquotes}
\usepackage{amsmath}

\graphicspath{ {./images/} }

\newcommand{\mylabnumber}{3}
\newcommand{\mylabtitle}{Исследование способов 
                            модульного тестирования программного обеспечения}
\newcommand{\mysubject}{Тестирование ПО}
\newcommand{\mylecturer}{Тлуховская Н.П.}

\renewcommand{\baselinestretch}{1.25} % Sets basic line stretch
\renewcommand{\headrulewidth}{0pt} % Remove horizontal line below header in fancyhdr

\addto\captionsrussian{
    \renewcommand{\figurename}{Рисунок} % Set a default picture caption
    \renewcommand{\tablename}{Таблица} % Set a default table caption
}

\captionsetup[table]{singlelinecheck=false} % To make a table caption appear left-aligned

\pagestyle{fancy}
\lhead{} \rhead{} \cfoot{} % Setting empty headers
\chead{\thepage} % Sets central header page numbering

\setlength{\parindent}{1.25cm}
\setlength{\parskip}{8pt}

 % Format section style and indentations
\titleformat{\section}[hang]{\large \centering \bfseries}{\thesection}{0.5em}{\MakeTextUppercase}
\titlespacing{\section}{\parindent}{1em}{0em}

% Format subsection style and indentations
\titleformat{\subsection}[hang]{\bfseries}{\thesubsection}{0.5em}{}
\titlespacing{\subsection}{\parindent}{1em}{0em}

% Format subsubsection style and indentations
\titleformat{\subsubsection}[hang]{\normalfont}{\thesubsubsection}{0.5em}{}
\titlespacing{\subsubsection}{\parindent}{1em}{0em}

% Format enumerate style with "enumitem" package
\setlist[enumerate, 1]{wide=\parindent,leftmargin=0pt,topsep=0pt,
    itemsep=0pt,partopsep=0pt,parsep=0pt}
\setlist[enumerate, 2]{wide=2\parindent,label=\alph*.,leftmargin=1.25cm,topsep=0pt,
    itemsep=0pt,partopsep=0pt,parsep=0pt}
\setlist[enumerate, 3]{wide=3\parindent,label=\alph*.,leftmargin=2.5cm,topsep=0pt,
    itemsep=0pt,partopsep=0pt,parsep=0pt}

% Format itemize style with "enumitem" package
\setlist[itemize, 1]{wide=\parindent,leftmargin=0pt,topsep=0pt,itemsep=0pt,
    partopsep=0pt,parsep=0pt}
\setlist[itemize, 2]{wide=2\parindent,leftmargin=1.25cm,topsep=0pt,itemsep=0pt,
    partopsep=0pt,parsep=0pt}
\setlist[itemize, 3]{wide=3\parindent,leftmargin=2.5cm,topsep=0pt,itemsep=0pt,
    partopsep=0pt,parsep=0pt}

\captionsetup[figure]{justification=centering}

\begin{document}

    \lstset{ % "listings package configuration"
        basicstyle=\footnotesize\ttfamily,
        breaklines=true,
        numbersep=5pt,
        tabsize=4,
        gobble=8,
        extendedchars=\true,
        keepspaces=\true,
        numbers=left,
        stringstyle=\ttfamily,
        showstringspaces=\false
    }

    % ############################################################################
    % -------------------------------- Title page --------------------------------
    % ############################################################################

    \begin{titlepage}
        
        \thispagestyle{empty}
        
        \begin{center}
            
            Министерство науки и высшего образования Российской Федерации \\
            Севастопольский государственный университет \\
            Кафедра ИС
            
            \vfill

            Отчет \\
            по лабораторной работе №\mylabnumber \\
            \enquote{\mylabtitle} \\
            по дисциплине \\
            \enquote{\MakeTextUppercase{\mysubject}}

        \end{center}

        \vspace{1cm}

        \noindent\hspace{7.5cm} Выполнил студент группы ИС/б-17-2-о \\
        \null\hspace{7.5cm} Горбенко К. Н. \\
        \null\hspace{7.5cm} Проверил \\
        \null\hspace{7.5cm} \mylecturer

        \vfill

        \begin{center}
            Севастополь \\
            2019
        \end{center}

    \end{titlepage}

    % ############################################################################
    % ------------------------------ Document start ------------------------------
    % ############################################################################

    \section{Цель работы}

    Исследовать основные подходы к модульному тестированию программного
    обеспечения. Приобрести практические навыки составления модульных тестов
    для объектно-ориентированных программ.

    \section{Постановка задачи}

    Выполнить модульное тестиование одного из классов, созданных в ходе ЛР № 1.
    Для этого:
    \begin{enumerate}
        \item Cоздать спецификацию тестового случая для одного из методов выбранного
              класса.
        \item Реализовать тестируемый класс и необходимое тестовое
              окружение на языке C\#.
        \item Выполнить тестирование с выводом результатов на экран и сохранением 
              в log-файл.
        \item Проанализировать результаты тестирования, сделать вывод.
    \end{enumerate}

    \section{Ход работы}

    Для модульного тестирования выберем класс ExtensionMethods:

    \begin{lstlisting}[language={[Sharp]C}]
        public static class ExtensionMethods
        {
            public static IEnumerable<int[]> GetColumns(this int[,] array)
            {
                if (array == null)
                {
                    throw new ArgumentNullException($"{nameof(array)} instance was null");
                }

                for (var j = 0; j < array.GetLength(1); j++)
                {
                    var column = new int[array.GetLength(0)];

                    for (var i = 0; i < array.GetLength(0); i++)
                        column[i] = array[i, j];

                    yield return column;
                }
            }
        }
    \end{lstlisting}

    Для реализации тестов создадим класс, представляющий один тестовый случай:

    \begin{lstlisting}[language={[Sharp]C}]
        public class Test
        {
            public string Name                 { get; set; }
            public int[,] Input                { get; set; }
            public IEnumerable<int[]> Expected { get; set; }
        }
    \end{lstlisting}

    Теперь создадим тестовое окружение:

    \begin{lstlisting}[language={[Sharp]C}]
        public class TestEngine
        {
            private ILogger           Logger    { get; set; }
            private IEnumerable<Test> TestSuite { get; set; }

            public TestEngine(ILogger logger, IEnumerable<Test> testSuite)
            {
                Logger    = logger    ?? throw new ArgumentNullException("Logger instance was null");
                TestSuite = testSuite ?? throw new ArgumentNullException("Test suite was null");
            }

            public void TestEquality()
            {
                foreach (var test in TestSuite)
                {
                    try
                    {
                        var actual = test.Input.GetColumns();
                        var result = actual.SequenceEqual(test.Expected, new Int32ArrayEqualityComparer());

                        Logger.Log($"Test {test.Name}," + 
                                $"result:{result}" + 
                                $"{(result ? "" : $",expected: {PrintArrayCollection(test.Expected)} actual:{PrintArrayCollection(actual)}")}");
                    }
                    catch (Exception e)
                    {
                        Logger.Log($"Test {test.Name}," +
                                $"result:{false}," +
                                $"Exception: {e.Message}");
                    }
                }
            }

            private string PrintArrayCollection(IEnumerable<int[]> collection)
            {
                var stringBuilder = new StringBuilder("{ ");

                foreach (var array in collection)
                {
                    stringBuilder.Append("[ ");
                    foreach (var item in array)
                    {
                        stringBuilder.Append($"{item} ");
                    }
                    stringBuilder.Append("] ");
                }
                stringBuilder.Append("}");
                return stringBuilder.ToString();
            }
        }
    \end{lstlisting}

    В этом классе:

    \begin{enumerate}
        \item Содержится поле, содержащее экземпляр файлового логгера.
        \item Содержится коллекция экземпляров тестов, представляющая
              набор тестовых случаев данного класса.
        \item Содержится метод, выполняющий непосредственно тестирование
              согласно спецификации, предоставленной набором тестовых случаев.
              При этом сравнивается ожидаемое значение метода с текущим.
        \item При провале теста для текущего тестового случая выводится ожидаемое
              значение результата выполнения метода и текущее значение для
              создания возможности проследить ошибки.
    \end{enumerate}

    Класс файл-логгера:

    \begin{lstlisting}[language={[Sharp]C}]
        public interface ILogger
        {
            void Log(string message);
        }
        
        public class FileLogger : ILogger
        {
            private readonly StreamWriter streamWriter;
            
            public FileLogger(StreamWriter writer)
            {
                streamWriter = writer ?? throw new ArgumentNullException("StreamWriter instance was null");
            }

            public void Log(string message)
            {
                if (message != null && !string.IsNullOrWhiteSpace(message))
                {
                    streamWriter.WriteLine(message);
                    streamWriter.Flush();
                }
            }
        }
    \end{lstlisting}

    Этот класс не позволяет записывать пустые сообщения в файл.

    Основная программа, запускающая тестовое окружение:

    \begin{lstlisting}[language={[Sharp]C}]
        public static class Program
        {
            static void Main(string[] args)
            {
                var fileLogger = new FileLogger(new StreamWriter(path, append:true));
                var testEngine = new TestEngine(fileLogger, TestSuite);
                testEngine.TestEquality();
            }

            private static string path = "file.txt";

            private static IEnumerable<Test> TestSuite = new[]
            {
                new Test {
                    Name     = "Single item",
                    Input    = new[,] { { 3 } },
                    Expected = new[] { new[] { 3 } }.AsEnumerable()
                },
                new Test {
                    Name = "Multiple items in square matrix",
                    Input = new[,]
                    {
                        { 15, 10, 36 },
                        { 23,  0, 14 },
                        { 62, 15, 35 }
                    },
                    Expected = new[]
                    {
                        new[] { 15, 23, 62 },
                        new[] { 10,  0, 15 },
                        new[] { 36, 14, 35 }
                    }.AsEnumerable()
                },
                new Test {
                    Name = "Multiple items in non-square matrix",
                    Input = new[,]
                    {
                        { 13,  2, 14, 16,  46 },
                        {  0, 54, -2,  9, -14 },
                        { 62,  0, 21, 15,   2 },
                        { 64,  0, 11,  2, 523 }
                    },
                    Expected = new[]
                    {
                        new[] { 13,   0, 62,  64 },
                        new[] {  2,  54, -2,   9 },
                        new[] { 14,  -2, 21,  11 },
                        new[] { 16,   9, 15,   2 },
                        new[] { 46, -14,  2, 523 }
                    }.AsEnumerable()
                },
            };
        }
    \end{lstlisting}

    В этом классе происходит создание экземпляров логгера и тестового окружения
    и запуск тестового окружения. При этом, класс соддержит коллекцию тестов,
    которую передает в тестовое окружение. 

    Результат работы программы(содержимое файла). В программе специально один из 
    тестовых примеров провален для демонстрации:

    \begin{lstlisting}[language={[Sharp]C}]
        Test Single item,result:True
        Test Multiple items in square matrix,result:True
        Test Multiple items in non-square matrix,result:False,expected: { [ 13 0 62 64 ] [ 2 54 -2 9 ] [ 14 -2 21 11 ] [ 16 9 15 2 ] [ 46 -14 2 523 ] } actual:{ [ 13 0 62 64 ] [ 2 54 0 0 ] [ 14 -2 21 11 ] [ 16 9 15 2 ] [ 46 -14 2 523 ] }
    \end{lstlisting}

    \section{Выводы}

    В ходе лабораторной работы было проведено модульное тестирование класса,
    включающее в себя создание тестового окружения. Результат тестирования был
    записан в файл. Преимуществом данного метода является возможность автоматизации
    процесса тестирования ПО. Недостатком является сложность реализации, при которой
    для каждого класса необходимо создавать тестовое окружение. Кроме того,
    при таком подходе существуют особенные сложности тестирования исключительных 
    случаев. 

\end{document}